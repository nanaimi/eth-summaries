\section*{Chapter 5: Approximation Theory}

\subsection*{Universality}

A function $f$ can be approximated by another class of functions $G$ $\to$ $\inf_{g\in G} d(f,g) = 0$.

$G$ univ. approx. iff $ C(s) \simeq G(s) \forall \text{ compact } S \subset \R^n$ 

\textbf{Weierstrass Thm}: Polynomials $\mathcal P$ are dense in $C(I)$, where $I = [a; b]$ for any $a < b$ (can approx all cont. functions in compacta)

- MLP with 1-hidden layer and smooth non-polynomial activation function $\sigma$ is a universal approximator for $C(\R)$

- Spans of ridge functions are universal approx.

\subsection*{Complexity}

How many units or parameters are required to obtain a desired approximation accuracy?

\textbf{Barron's thm}: for some well-behaved functions (fulfill grad. regularity), MLPs with sigmoidal activation and $m$-layers: do not suffer from the curse of dimensionality and error bounded by $\mathcal O(\frac 1 m)$ $\to$ are expressive and efficient.

\textbf{Thm}: there exits a function $g$ s.t. $g$ is expressible by a 2 hidden layer network with width polynomial in $n$ but one hidden layer needs exponentially many units in $n$. $\to$ in some cases, depth provides exponential benefit when approx. functions.